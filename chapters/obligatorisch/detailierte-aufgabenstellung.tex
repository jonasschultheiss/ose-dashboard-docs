\section{Detaillierte Aufgabenstellung}

Die bestehende Web Anwendung „OSE Dashboard“ zeigt das One Story Exhibit Model (OSE Modell), welches in Reinach steht, in einer 3D Ansicht an. Darin werden die Daten der Messgeräte aus der Netilion Cloud gelesen und ebenfalls dargestellt.
\newline
Das Ziel dieses Projekts ist, dass die Web Anwendung auch für andere OSE Modelle, welche dem gleichen Aufbau haben, verwendet werden kann, ohne dass zukünftig eine Änderung an der Applikation notwendig ist.
\newline
Die bestehende Anwendung muss dafür so erweitert werden, dass Verlinkung des 3D Modells mit den Daten der Messgeräte nicht mehr im Source Code hinterlegt ist. Es muss ein Weg gefunden werden, diese Verlinkung für alle bestehenden und zukünftigen OSE Modelle zu speichern.
\newline
Laut Autraggeber haben alle OSE Modelle den gleichen Aufbau mit Messgeräten vom gleichen Typ. Auch die Bezeichnung der Messgeräte sollte bei allen Modellen gleich sein. Wird die Anwendung das erste mal für ein OSE Modell verwendet, muss für alle Geräte aus diesem OSE Modell die Verlinkung mit dem 3D Modell der Anwendung automatisch erfolgen und gespeichert werden. Sollte es aber Messgeräte geben, die nicht vom gleichen Typ sind, weil sie z.B. durch eine neuere Version ersetzt wurden, dann soll der Anwender die Möglichkeit haben über ein Konfigurationsmenü die Verlinkung vorzunehmen.
\newline
Änderungen an der Konfiguration dürfen nicht von jedem User vorgenommen werden. Das Konfigurationsmenü darf nur von Usern geöffnet werden, welche die entsprechende Berechtigung haben. Dafür soll in Netilion eine User Gruppe erstellt werden. Alle User dieser Gruppe dürfen dann die Konfiguration ändern.
\newline
Die Anwendung selber soll aber weiterhin ohne Login aufrufbar sein. Der Anwender soll als Datenquelle für das 3D Modell zwischen den verschiedenen integrierten Standorten wählen können. Da jedes OSE Modell einen eigenen User hat, müssen die User Credentials der einzelnen OSE Modelle ebenfalls in der Applikation gespeichert werden. Dabei muss darauf geachtet werden, dass diese Daten sicher gespeichert werden und der Anwender keinen Zugriff darauf hat.
\newline
Die Methoden mit der Logik für die automatische Verlinkung der Messgeräte mit dem 3D Modell soll automatisiert getestet werden.
\newline
Für den Test der kompletten Applikation sind manuelle Tests ausreichend. Dabei sind folgende Testsfälle zu beachten:
\begin{itemize}
  \item Konfigurationsmenu nur mit entsprechender Berechtigung aufrufbar
  \item Alle Messgeräte werden automatisch verlinkt.
  \item Messgeräte können nicht automatisch verlinkt werden .
  \item User kann ohne Login zwischen integrierten OSE Modellen wechseln.
  \item Credentials der OSE Modelle sind sicher gespeichert
\end{itemize}
Die Tests sollen zuerst an dem OSE Modell in Reinach durchgeführt werden. Bei diesem Modell können auch für Tests die Daten in Netilion mal abgeändert werden.
Zum Abschluss sollen 2 weitere OSE Modelle integriert werden um die funktionalität der Anwendung mit mehreren OSE Modellen zu testen.
